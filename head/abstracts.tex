%\begingroup
%\let\cleardoublepage\clearpage


% English abstract
\cleardoublepage
\chapter*{Abstract}
\markboth{Abstract}{Abstract}
\addcontentsline{toc}{chapter}{Abstract (English/Français)} % adds an entry to the table of contents
% put your text here
Stellarators are 3-dimensional (3D) magnetic fusion confinement devices. In 3D, magnetic equilibria are in general composed of nested magnetic surfaces, magnetic islands and chaotic field lines, where the latter two topologies are detrimental to core confinement. Consequently, configurations with large regions occupied by nested magnetic surfaces are usually sought. In general, it is possible to design stellarators with nested magnetic surfaces in vacuum; at larger plasma $\beta$ however, the currents generated by the plasma such as the diamagnetic, Pfirsch-Schl¨uter or bootstrap current, produce perturbations in the magnetic field, and ultimately destroy the carefully designed magnetic surfaces. This process is thought to set the maximum achievable $\beta$ in stellarators, in opposition to tokamaks in which the plasma stability usually sets this limit. To date, there is no theory, nor extensive numerical study that characterizes the equilibrium $\beta$-limit in stellarators, nor a theory on its dependency on other relevant operational parameters. In this work, we study equilibrium $\beta$-limits in a rotating ellipse. We use the Stepped Pressure Equilibrium Code (SPEC), which can compute 3D magnetic equilibria with magnetic islands and chaotic field lines in a reasonable amount of time. SPEC has been recently extended to allow the computation of free-boundary equilibria with prescribed net toroidal current profile. Leveraging these new capabilities, we perform large parameter scans and discuss how changes in the topology of magnetic field lines can be detected. Two kinds of equilibrium $\beta$-limit are identified depending on the bootstrap current strength; one where a central magnetic island appears and grows with $\beta$, and another one above which a region occupied by chaotic field lines emerges. Interestingly, we will show that in some considered stellarator geometries, a small bootstrap current can be beneficial and increases the equilibrium $\beta$-limit. An analytical model based on the High Beta Stellarator (HBS) expansion will be proposed to explain our numerical results and to expose the underlying scaling laws. Finally, by coupling SPEC with the optimization framework SIMSOPT, we optimize free-boundary equilibria at finite $\beta$ to reduce the volume of plasma occupied by chaotic magnetic field lines and retrieve nested magnetic surfaces. Three optimizations are considered, where either (i) the plasma boundary, or (ii) the coil geometry and currents, or (iii) the injected toroidal current profile are optimized.

% French abstract
\begin{otherlanguage}{french}
\cleardoublepage
\chapter*{Résumé}
\markboth{Résumé}{Résumé}
% put your text here
\lipsum[1-2]
\end{otherlanguage}


%\endgroup			
%\vfill
